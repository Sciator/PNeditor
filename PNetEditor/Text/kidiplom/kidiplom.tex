%%%  Ukázkový text a dokumentace stylu pro text závěrečné (bakalářské a
%%%  diplomové) práce na KI PřF UP v Olomouci
%%%  Copyright (C) 2012 Martin Rotter, <rotter.martinos@gmail.com>
%%%  Copyright (C) 2014 Jan Outrata, <jan.outrata@upol.cz>


%%  Pro získání PDF souboru dokumentu je třeba tento zdrojový text v
%%  LaTeXu přeložit (dvakrát) programem pdfLaTeX.

%%  V případě použití programu BibLaTeX pro tvorbu seznamu literatury
%%  je poté ještě třeba spustit program Biber s parametrem jméno
%%  souboru zdrojového textu bez přípony a následně opět (dvakrát)
%%  přeložit zdrojový text programem pdfLaTeX.

%%  Postup získání Postscriptového souboru je popsán v dokumentaci.


%%  Třída dokumentu implementující styl pro závěrečnou práci. Vybrané
%%  nepovinné parametry (ostatní v dokumentaci):

%%  'master' pro sazbu diplomové práce, jinak se sází bakalářská práce

%%  'field=kód' pro Váš studijní obor, kódy pro diplomovou práci 'uvt'
%%  pro Učitelství výpočetní techniky pro střední školy a 'binf' pro
%%  Bioinformatiku, jinak je výchozí Informatika, a pro bakalářskou
%%  práci 'ainfk' pro Aplikovanou informatiku v kombinované formě,
%%  'inf' pro Informatiku, 'infv' pro Informatiku pro vzdělávání a
%%  'binf' pro Bioinfomatiku, jinak je výchozí Aplikovaná informatika
%%  v prezenční formě

%%  'printversion' pro sazbu verze pro tisk (nebarevné logo a odkazy,
%%  odkazy s uvedením adresy za odkazem, ne odkazy do rejstříku),
%%  jinak verze pro prohlížeč

%%  'biblatex' pro zapnutí podpory pro sazbu bibliografie pomocí
%%  BibLaTeXu, jinak je výchozí sazba v prostředí thebibliography

%%  'language=jazyk' pro jazyk práce, jazyky english pro anglický,
%%  slovak pro slovenský, jinak je výchozí czech pro český

%%  'font=sans' pro bezpatkový font (Iwona Light), jinak výchozí
%%  patkový (Latin Modern)

\documentclass[
%  master,
%  field=inf,
%  printversion,
  biblatex,
%  language=english,
%  font=sans,
  glossaries,
  index
]{kidiplom}

\usepackage{algorithm,algorithmicx,algpseudocode}

%% Informace pro úvodní strany. V jazyku práce (pokud není v komentáři
%% uvedeno česky) a anglicky. Uveďte všechny, u kterých není v
%% komentáři uvedeno, že jsou volitelné. Při neuvedení se použijí
%% výchozí texty. Text pro jiný než nastavený jazyk práce (nepovinným
%% parametrem language makra \documentclass, výchozí český) se zadává
%% použitím makra s uvedením jazyka jako nepovinného parametru.

%% Název práce, česky a anglicky. Měl by se vysázet na jeden řádek.
\title{Editor Petriho Sítí}
\title[english]{Petri Nets Editor}

%% Volitelný podnázev práce, česky a anglicky. Měl by se vysázet na
%% jeden řádek. Výchozí je prázdný.
\iffalse
\subtitle{Ukázkový text a dokumentace stylu v \LaTeX{}u}
\subtitle[english]{Sample text and documentation of the \LaTeX{} style}
\fi

%% Jméno autora práce. Makro nemá nepovinný parametr pro uvedení
%% jazyka.
\author{Roman Wehmhöner}

%% Jméno vedoucího práce (včetně titulů). Makro nemá nepovinný
%% parametr pro uvedení jazyka.
\supervisor{Mgr. Petr Osička, Ph.D.}

%% Volitelný rok odevzdání práce. Výchozí je aktuální (kalendářní)
%% rok. Makro nemá nepovinný parametr pro uvedení jazyka.
%\yearofsubmit{\the\year}

%% Anotace práce, včetně anglické (obvykle překlad z jazyka
%% práce). Jeden odstavec!
\annotation{Cílem bakalářské práce bylo vytvořit editor 
  petriho sítí umožňující jednoduché a pohodlné ovládání. 
  Editor také obsahuje  }

\annotation[english]{Sample text of thesis at the \kitextdepten,
  \kitextfacultyen, \kitextuniven{} and, at the same time,
  documentation of the \LaTeX{} style for the text. The source text in
  \LaTeX{} is recommended to be used as a template for real student's
  thesis text.}

%% Klíčová slova práce, včetně anglických. Oddělená (obvykle) středníkem.
\keywords{styl textu; závěrečná práce; dokumentace; ukázkový text}
\keywords[english]{text style; thesis; documentation; sample text}

%% Volitelná specifikace příloh textu práce, i anglicky. Výchozí je '1
%% CD/DVD'.
%\supplements{jedno kulaté placaté CD/DVD s malou kulatou dírou uprostřed}
%\supplements[english]{one round flat CD/DVD with a small round hole in the middle}

%% Volitelné poděkování. Stručné! Výchozí je prázdné. Makro nemá
%% nepovinný parametr pro uvedení jazyka.
\thanks{Děkuji, děkuji, děkuji.}

%% Cesta k souboru s bibliografií pro její sazbu pomocí BibLaTeXu
%% (zvolenou nepovinným parametrem biblatex makra
%% \documentclass). Použijte pouze při této sazbě, ne při (výchozí)
%% sazbě v prostředí thebibliography.
\bibliography{bibliografie.bib}

%% Další dodatečné styly (balíky) potřebné pro sazbu vlastního textu
%% práce.
\usepackage{lipsum}

\setlength{\parskip}{1.2em}

\begin{document}
%% Sazba úvodních stran -- titulní, s bibliografickými údaji, s
%% anotací a klíčovými slovy, s poděkováním a prohlášením, s obsahem a
%% se seznamy obrázků, tabulek, vět a zdrojových kódů (pokud jejich
%% sazba není vypnutá).
\maketitle

%% Vlastní text závěrečné práce. Pro povinné závěry, před přílohami,
%% použijte prostředí kiconclusions. Povinná je i příloha s obsahem
%% přiloženého CD/DVD.

%% -------------------------------------------------------------------

\newcommand{\BibLaTeX}{\textsc{Bib}\LaTeX}



















\newcommand{\todo}[1]{\textcolor{red}{TODO: #1}\PackageWarning{TODO:}{#1!}}
\todo{Smazat todo: command}

\section{Petriho sítě}
Táto kapitola byla inspirovaná a čerpala informace z Understanding petri nets\cite{reisig2013understanding}

\subsection{Základní definice}

Petriho síťe jsou matematickým nástrojem 
pro modelování a simulaci paralelních procesů a jejich sychronizaci.

$$ N = \langle P,T,A,M_{0}\rangle $$
\begin{itemize}
  \item $N$ je Petriho sítí
  \item $P$ je konečná množina míst
  \item $T$ je konečná množina přechodů
  \item $A$ je konečná množina hran
    $ A \subseteq ((P \times T) \cup (T \times P)) \times \mathbb{N}_0 $ \\
    kde číslo symbolizuje násobek kolik značek hrana 'přesune'
  \item $M_{0} : P \rightarrow \mathbb{N}_0$ je počáteční ohodnocení sítě (zkráceně ohodnocení) míst 
    kde pro každé místo $p \in P$ existuje počet jeho značek $m \in M_{0}$ 
\end{itemize}

Pro odkazovaní na jednotlivé členy prvků z množiny hran budeme používat 
  notaci $P(a)$ pro odkázání na místo v hraně $a \in A$, 
  $T(a)$ pro odkázání na přechod a $AM(a)$ pro odkaz na násobek.

Každý přechod $t$ může mít 'přiřazený' libovolný počet 
  hran, kde každá hrana $a$ je propojením s některým z míst $p \in P$.\\
Hrany přechodu $t$ můžeme rozlišit na hrany směrující do přechodu 
  $$^\circ t = \{a \in A | a \in (P \times T \times \mathbb{N}_0) \land t = T(a)\}$$
a hrany směřující do místa 
  $$ t ^\circ  = \{a \in A | a \in (T \times P \times \mathbb{N}_0) \land t = T(a)\}$$
dohromady pak všechny hrany přechodu $t$ jsou spojením těchto dvou množin
  $$ArcesOfTransition(t,A) = (^\circ t \cup t ^\circ) \subset A$$
Aktuální stav petriho sítě neboli ohodnocení M je funkce přiřazující každému
 místu $p \in P$ petriho sítě počet značek 
  $$M(p) \in \mathbb{N}_0$$
Výchozí ohodnocení petriho sítě se značí $M_{0}$.

Pro dané ohodnocení $M$ je přechod $t \in T$ označený jako povolený 
  pokud všechny hrany směřující do přechodu $\,^\circ t$ splňují svou podmínku tzn.
  hrana splňuje svoji podmínku pokud místo ze kterého vychází má vyšší nebo stejné ohodnocení 
  (v daném $M$) než je násobek hrany $AM$
$$IsEnabled(P,t,A,M) = (\forall a \in \,^\circ t)M(P(a)) \geq AM(a)$$
Pak si můžeme ještě definovat množinu všech povolených přechodů pro zadané ohodnocení
$$
 EnabledTransitions(P,T,A,M) = 
 \{ t \in T | IsEnabled(P,t,A,M) \}
$$


\todo{testovací hrany}


\subsection{Vizuální zobrazení sítě}
\todo{obrázky}


\subsection{Využití}

\subsection{Graf dosa}

\subsection{Graf pokrytí}

\subsubsection{Sestrojení grafu}

\todo{definice a úprava EnabledTransitions}

\begin{algorithm}
 \caption{MakeCoverabilityGraph}
 \begin{algorithmic}[1]
  \Function{MakeCoverabilityGraph}{$\langle P,T,A,M_0\rangle$}
    \State $\langle V,E,v_0\rangle := \langle\{M_0\},\emptyset,M_0\rangle$
    \State $WorkSet := \emptyset $
    \ForAll{$T \in EnabledTransitions(P,T,A,M_0)$}
      \State $WorkSet := WorkSet \cup \{\langle M_0, T \rangle\} $
    \EndFor

    \While{$WorkSet \neq \emptyset$}
      \State $\langle M, t \rangle := RandomElement(WorkSet)$
      \State $WorkSet := WorkSet \setminus \{\langle M, t \rangle\}$
      \State $M' := FireTransition(M,t)$
      \ForAll{$\{M'' \;|\; M'' \in V \land (M'' \to_p M \lor M'' = M) \land M'' < M'\}$} 
        \\ \Comment{$a \to_p b = \text{path from a to b}$}
        \ForAll{$p \in P$}
          \State $mp := M(p)$
          \If{$M''(p)<M'(p)$}
            \State $M'(p) := \omega$
          \EndIf
        \EndFor
      \EndFor

      \If{$M' \notin V$}
        \State $V := V \cup \{M'\}$
        \ForAll{$T \in EnabledTransitions(P,T,A,M')$}
          \State $WorkSet := WorkSet \cup \{\langle M', T \rangle\} $
        \EndFor
      \EndIf
      \State $E := E \cup \{\langle M,t,M'\rangle\}$
    \EndWhile

    \State \textbf{return} $\langle V,E,v_0\rangle$
  \EndFunction
 \end{algorithmic}
\end{algorithm}


  
\section{Implementované algoritmy}

\subsection{Příklady sítí}
\todo{síť + analýza}
  




\section{Editor}
\subsection{Systémové požadavky}
\todo{vyžaduje myš}
\subsection{Ovládání}
\todo{obrázek rozložení editoru}
\subsubsection{Postranní panel}
\todo{jiné pojmenování ?}
\subsubsection{Hlavní plocha editoru}
\subsubsection{Panel nástrojů editoru}
\subsubsection{Tabulka ohodnocení}
\subsubsection{Výsledky analýzy}
\subsubsection{Tisk sítě}
\todo{obrázek jak vytisknotu do PDF}

\subsubsection{Klávesové zkratky}





\section{Použité technologie}

\subsection{nodejs}
Aplikace je psaná za pomoci opensourcové technologie nodeJS, která umožňuje využívat jazyk 
JavaScript pro psaní serverových aplikací. Cílem platformy nodeJS je vytvořit
ekosystém pro jednoduší vývoj webových stránek a aplikací kde stačí pro vyrvaření 
funkcionality pouze jeden programovací jazyk.

\subsection{Typescript}
Typescript je opensource programovací jazyk od společnosti Microsoft který je nadstavbou nad jazykem JavaScript.
Jelikož je Typescript nadstavbou nad programovacím jazykem JavaScript tak je jakýkoliv validní kód v JavaScriptu automaticky validním kódem v Typescriptu.
Typecript se kompiluje do Javascriptu a proto po stránce funkcionality nenabízí nic navíc avšak po stránce vývoje nabízí možnost statické typové kontroly 
se kterou je spjaté fungování našeptávačů v dnešních textových editorech a také nabízí možnost kompilace do starších verzí JavaScript se simulací funkcionality novejších verzí JavaScriptu.
\todo{příklady kódu ?}


\subsection{Electron}
Electron je opensource framework pro vytváření desktopových aplikací pomocí webových technologií JavaScript, HTML a CSS. 
Využívá NodeJS


\subsection{Javascriptová Knihovna Data driven documents (D3)}
\todo{příklady kódu ?}

\subsection{Scalable Vector Graphics (SVG)}
\todo{příklady kódu ?}




\section{Stavba programu}

\subsection{Třída 1}
\subsection{Třída 2}
\subsection{Třída 3}
\subsection{Třída 4}



































\todo{pokračování ukázkového textu}

\noindent\textcolor{red}{\LARGE Upozornění: Následující text
  dokumentace stylu, vyjma přílohy~\ref{sec:ObsahCD}, je rozpracovaná
  a (značně) neúplná verze!!!}

\section{Styly pro psaní bakalářských a diplomových prací}
Toto jsou styly pro psaní bakalářských a diplomových prací přes typografický systém \LaTeX{}, tedy \textbf{kistyles}.

\subsection{Požadavky a podprovaná prostředí}
Sada balíku \textbf{kistyles} podporuje následující distribuce systému \LaTeX{}:
\begin{itemize}
\item \TeX{} Live.
\end{itemize}

Jsou podporovány všechny výstupní ovladače, tedy jak \textbf{dvi}, tak \textbf{pdf} i \textbf{ps}. Funkčnost zmiňovaných distribucí byla ověřena na několika operačních systémech, mezi které patří:
\begin{enumerate}
\item Windows $8.1$,
\item Archlinux,
\item Debian.
\end{enumerate}

Důrazně se doporučuje používat aktuální verzi dané distribuce systému \LaTeX{}.

%%%  Po přeložení programem CSLaTeX (třikrát) je potřeba použít
%%%  program DVIPS a takto získaný PostScriptový soubor vytisknout
%%%  na PostScriptové tiskárně nebo pomocí programu GhostScript.
%%%
%%%  Rovněž je možné použít program DVIPDFM a vytvořit z dokumentu
%%%  soubor ve formátu PDF včetně hypertextových odkazů.

\subsection{Přepínače}
Styl kidiplom je z hlediska uživatele zastoupen ekvivalentně nazvanou třídou, kterou je třeba volat na záčátku dokumentu:
\begin{kicode}{TeX}{}{Volání třídy \textbf{kidiplom}}
\documentclass[
  master=true,
  font=sans,
  printversion=false,
  joinlists=true,
  glossaries=true,
  figures=true,
  tables=true,
  sourcecodes=true,
  theorems=true,
  bibencoding=utf8,
  language=czech,
  encoding=utf8,
  field=inf,
  index=true,
  biblatex=true
]{kidiplom}
\end{kicode}

Následuje přehled přepínačů, je vždy uvedeno jméno přepínač, včetně výchozí hodnoty. Přepínače uvádí tabulka \ref{tab:prepinace}.

\begin{table}
\begin{center}
\caption{Seznam přepínačů}\label{tab:prepinace}
\scalebox{0.95}{\begin{tabular}{>{\bfseries}l >{\ttfamily}c L{8cm}}
{\normalfont Přepínač} & {\normalfont Výchozí hodnota} & {\normalfont Popis} \\
\hline
master & false & Povolí nebo zakáže režim diplomové práce. Výchozí režim je tedy bakalářská práce. \\

field & ainfp & Specifikuje studijní obor:\newline
\begin{description}
\item[ainf] Aplikovaná informatika\,--\,prezenční,
\item[ainfk] Aplikovaná informatika\,--\,kombinovaná,
\item[inf] Informatika\,--\,prezenční,
\item[infv] Informatika ve vzdělávání\,--\,kombinovaná,
\item[binf] Bioinformatika\,--\,prezenční.
\end{description} \\

font & serif & Zapne či vypne podporu pěkného bezpatkového fontu. Možné hodnoty jsou:\newline
\begin{description}
\item[sans] Bezpatkové písmo (písmo Iwona).
\item[serif] Patkové písmo (písmo Computer Modern).
\end{description} \\

%%  'encoding=kódování' pro kódování tohoto a vložených zdrojových
%%  textů v kódování jiném než výchozím utf8
encoding & utf8 & Kódování souboru dokumentu, doporučuje se ponechat výchozí hodnotu. \\

bibencoding & utf8 & Kódování souboru bibliografie. Tato volba má smysl pouze, pokud je použita bibliografie skrze balíček \BibLaTeX{}. \\

language & czech & Jazyk práce. \\

printversion & false & Je-li zapnuto, pak budou odkazy vysázeny optimalizovaně pro knižní sazbu. Tuto volbu je nutno použít pro tisk práce. \\

%%% Nepovinné argumenty `tables' a `figures' použijte pouze v případě,
%%% že váš dokument obsahuje tabulky a obrázky a chcete vytvořit
%%% jejich seznamy za obsahem.
%%%
%%% Argument `joinlists' způsobí zřetězení obsahu a seznamů tabulek a obrázků.
%%% Není-li použít, všechny seznamy jsou uvedeny na samostatných stránkách.

joinlists & true & Je-li zapnuto, pak seznamy obrázků, tabulek, vět a
zdrojových kódů sázené za obsahem nebudou rozděleny na samostatné stránky. \\

figures & true & Je-li zapnuto, pak v seznamech položek bude zahrnut seznam obrázků. \\

tables & true & Je-li zapnuto, pak v seznamech položek bude zahrnut seznam tabulek. \\

theorems & false & Je-li zapnuto, pak v seznamech bude zahrnut seznam teorémů. \\

sourcecodes & false & Je-li zapnuto, pak v seznamech bude zahrnut seznam zdrojových kódů. \\

glossaries & false & Je-li zapnuto, pak na konci dokumentu bude vysázen seznam zkratek. \\

index & false & Zapíná podporu sazby rejstříku. \\

biblatex & true & Zapne sazbu bibliografie přes balík \BibLaTeX{}.
\end{tabular}}
\end{center}
\end{table}

\subsection{Geometrie stránky}
Tento styl používá list velikosti $A4$. Pro sazbu prací je třeba použít jednostrannou sazbu. Levý okraj je rozšířen s ohledem na vazbu výsledné knižní podoby práce.









\section{Sazba částí dokumentu}
\subsection{Sazba úvodní strany či obsahu}
Vysázení všech podstatných částí úvodu práce obstará makro \kiinlinecode{TeX}{!}{\\maketitle}. Pro správné vysázení všech částí a meta-informací je potřeba použí makra \kiinlinecode{TeX}{!}{\\title}, \kiinlinecode{TeX}{!}{\\author} a další. Jejich přehled lze najít ve zdrojovém souboru tohoto dokumentu. V případě použítí \textbf{pdf} výstupu se generuje i dodatečná hlavička souboru s meta-informacemi jako je autor dokumentu, název práce či dalšími.

\subsection{Závěry}
Závěr práce by se měl poskytnout jak v původním jazyce práce, tak v jazyce anglickém. Pro sazbu závěru jsou k dispozici příslušná makra. Berte na vědomí, že v anglickém závěru se aktivuje plně anglická sazba se všemi konvencemi. Tedy je třeba používat anglické uvozovky a další správné typografické prvky.

\begin{kicode}{TeX}{}{Sazba závěrů}
% Tiskne český závěr práce.
\begin{kiconclusions}
Závěr práce v \uv{českém} jazyce.
\end{kiconclusions}

% Tiskne anglický závěr práce.
\begin{kiconclusions}[english]
Thesis conclusions written in \uv{English}.
\end{kiconclusions}
\end{kicode}

\subsection{Matematika}
Pro sazbu matematiky je k dispozici sada standardních maker.
$$\langle f \rangle, \lfloor g \rfloor,
\lceil h \rceil, \ulcorner i \urcorner$$

$$\left\{\frac{x^2}{y^3}\right\}$$

$$
A_{m,n} =
 \begin{pmatrix}
  a_{1,1} & a_{1,2} & \cdots & a_{1,n} \\
  a_{2,1} & a_{2,2} & \cdots & a_{2,n} \\
  \vdots  & \vdots  & \ddots & \vdots  \\
  a_{m,1} & a_{m,2} & \cdots & a_{m,n}
 \end{pmatrix}
$$

$$
M = \begin{bmatrix}
       \frac{5}{6} & \frac{1}{6} & 0           \\[0.3em]
       \frac{5}{6} & 0           & \frac{1}{6} \\[0.3em]
       0           & \frac{5}{6} & \frac{1}{6}
     \end{bmatrix}
$$

\subsection{Sazba literatury}
Pro sazbu literatury má uživatel dvě možnosti. Může použít služeb balíků \BibLaTeX{}, který je pro \textbf{kistyles} zapnutý, či lze použít manuální sazbu bibliografie.
\subsubsection{Sazba bibliografie přes \BibLaTeX{}}
Při použití tohoto balíku se data o použité literatuře ukládají do dedikovaného textového souboru, ukázku najdete i v tomto stylu pod jménem \kiinlinecode{text}{!}{bibliografie.bib}.

Formát daného souboru je nad rámec této dokumentace a je na každém uživateli, aby si jej nastudoval. Bibliografie se tiskne makrem \kiinlinecode{TeX}{!}{\\printbibliography}. Taktéž v preambuli dokumentu je třeba definovat, který soubor data bibliografie obsahuje, tedy například \kiinlinecode{TeX}{!}{\\bibliography\{bibliografie.bib\}}.

Dokument, který využívá \BibLaTeX{} je následně nutné přeložit jak pomocí překladače zvoleného ovladače, tak pomocí aplikace \kiinlinecode{text}{!}{biber}. Více informací poskytne soubor \kiinlinecode{text}{!}{Makefile} z distribuce tohoto stylu.

Výhodou tohoto přístupu je, že bibliografie se vysází automaticky a (obvykle) není třeba manuální úprava formátování.

\subsubsection{Manuální sazba bibliografie}
Manuální sazba obnáší vysázení prostředí \kiinlinecode{text}{!}{thebibliography} ručně. To je nad rámec tohoto dokumentu. Ukázku tohoto přístupu lze samozřejmě nalézt ve zdrojovém souboru tohoto dokumentu nebo také \href{http://www.math.uiuc.edu/~hildebr/tex/bibliographies.html}{zde}.

Pro aktivaci manuální sazby bibliografie je třeba volat třídu \kiinlinecode{text}{!}{kidiplom} s parametrem \kiinlinecode{text}{!}{biblatex=false}. Mějte, prosím, na paměti, že v tomto módu jsou makra \kiinlinecode{text}{!}{\\bibliography} a \kiinlinecode{text}{!}{\\printbibliography} nedostupná.

\subsection{Drobná makra}
Základní styl definuje hned několik maker pro usnadnění práce. Například makro \kiinlinecode{TeX}{!}{\\buno} vysází řetezec \uv{bez újmy na obecnosti}. Je k dispozici i verze s prvním velkým písmenem, \kiinlinecode{TeX}{!}{\\Buno}.

Je rovněž možno přidávat položky do seznamu zkratek. K tomu slouží makro \kiinlinecode{TeX}{!}{\\newacronym}, které lze použít například jednoduše jako \kiinlinecode{TeX}{!}{\\newacronym\{UPOL\}\{UPOL\}\{\\kitextunivcz\}}. Na danou zkratku se pak lze odkazovat jednoduše, \kiinlinecode{TeX}{!}{\\gls\{UPOL\}}.

Sazba uvozovek respektuje nastavení částí dokumentu, a proto se doporučuje používat makro \kiinlinecode{TeX}{!}{\\uv}. V anglické závěru práce toto platí taky, viz tato PDF ukázka.

Styl podporuje sazbu odstavců v tabulkách, více obsahuje tabulka \ref{tab:odstavce}.

\begin{table}
\begin{center}
\caption{Seznam přepínačů}\label{tab:odstavce}
\begin{tabular}{L{4cm}|R{4cm}|L{4cm}}
\lipsum[23] & \lipsum[22] & \lipsum[21]
\end{tabular}
\end{center}
\end{table}

K dispozici jsou také makra pro sazbu \csharp{} (\kiinlinecode{TeX}{!}{\\csharp}) či \cpp{} (\kiinlinecode{TeX}{!}{\\cpp}).

%% v případě tvorby rejstříku přeložit vygenerovaný soubor .idx
%% programem Makeindex a v případě tvorby seznamu zkratek spustit
%% program Makeglossaries s parametrem jméno souboru zdrojového textu
%% bez přípony a následně opět (dvakrát) přeložit zdrojový text
%% programem pdfLaTeX.

\subsection{Sazba rejstříku}
Sazba rejstříku sestává z několika kroků:

\begin{enumerate}
\item Je třeba přes volbu \kiinlinecode{TeX}{!}{index=true} rejstříkování povolit.
\item Použítím makra \kiinlinecode{TeX}{!}{\\index} rejstříkovat vybrané pojmy.
\item Kompilovat s použitím utility \kiinlinecode{TeX}{!}{makeindex}. Pro specifika tohoto kroku si stačí prohlédnout soubor \kiinlinecode{text}{!}{Makefile}.
\end{enumerate}

Makro \kiinlinecode{TeX}{!}{\\index} je redefinováno tak, že sází klikací odkaz na výraz v rejstříku. Je doporučeno jej použít ihned za výrazem\index{výraz}.

\textbf{Omezení redefinovaného makra \kiinlinecode{TeX}{!}{\\index}}: klikací odkaz nefunguje, pokud použijete konstrukci \kiinlinecode{TeX}{!}{\\index\{výraz|makro\}} (resp. \kiinlinecode{TeX}{!}{\\index\{výraz|(makro\}}), např. \kiinlinecode{TeX}{!}{\\index\{výraz|textit\}}.

Rejstřík lze vysázet pomocí makra \kiinlinecode{TeX}{!}{\\printindex}.

\subsection{Sazba zdrojových kódů}
Styl nabízí dva způsoby sazby zdrojových kódů:

\begin{enumerate}
\item Sazbu řádkových kódů, například \kiinlinecode{CSS}{!}{background-color: white;}. K tomu slouží makro formátu \kiinlinecode{TeX}{!}{\\kiinlinecode\{jazyk\}\{separátor\}\{kód\}}. Za separátor je vhodné volit jakýkoliv znak, který se nevyskytuje v samotném sázeném zdrojovém kódu. Za jazyk je nutno dosadit jeden z těchto: C, TeX, PHP, HTML, Lisp, SQL, TeX, Python, Java, TutorialD, text, csharp, cpp, JavaScript, CSS.

\item Sazbu zdrojových kódu do separátních prostředí. Takto vytištěný kód se objeví v seznamu zdrojových kódů. Ukázka například zdrojový kód \ref{kod:cpp}. Ukázku sazby naleznete ve zdrojovém kódu tohoto dokumentu.
\end{enumerate}

\newacronym{UPOL}{UPOL}{\kitextunivcz}

\begin{definition}[Název definice]
Abcd. Abcd. Abcd. Abcd. Abcd. Abcd. Abcd. Abcd. Abcd. Abcd. Abcd. Abcd. Abcd. Abcd. Abcd. Abcd. Abcd. Abcd. Abcd. Abcd. Abcd. Abcd. Abcd. Abcd. Abcd. Abcd. Abcd. Abcd. Abcd. Abcd. \gls{UPOL}
\end{definition}

\begin{proof}[Název důkazu]
Abcd. Abcd. Abcd. Abcd. Abcd. Abcd. Abcd. Abcd. Abcd. Abcd. Abcd. Abcd. Abcd. Abcd. Abcd. Abcd. Abcd. Abcd. Abcd. Abcd. Abcd. Abcd. Abcd. Abcd. Abcd. Abcd. Abcd. Abcd. Abcd. Abcd. 
\end{proof}

\begin{remark}[Pumpovací věta]
Abcd. Abcd. Abcd. Abcd. Abcd. Abcd. Abcd. Abcd. Abcd. Abcd. Abcd. Abcd. Abcd. Abcd. Abcd. Abcd. Abcd. Abcd. Abcd. Abcd. Abcd. Abcd. Abcd. Abcd. Abcd. Abcd. Abcd. Abcd. Abcd. Abcd. 
\end{remark}

\begin{example}[Pumpovací věta]
Abcd. Abcd. Abcd. Abcd. Abcd. Abcd. Abcd. Abcd. Abcd. Abcd. Abcd. Abcd. Abcd. Abcd. Abcd. Abcd. Abcd. Abcd. Abcd. Abcd. Abcd. Abcd. Abcd. Abcd. Abcd. Abcd. Abcd. Abcd. Abcd. Abcd. 
\end{example}

\begin{lemma}[Název definice]
Abcd. Abcd. Abcd. Abcd. Abcd. Abcd. Abcd. Abcd. Abcd. Abcd. Abcd. Abcd. Abcd. Abcd. Abcd. Abcd. Abcd. Abcd. Abcd. Abcd. Abcd. Abcd. Abcd. Abcd. Abcd. Abcd. Abcd. Abcd. Abcd. Abcd. 
\end{lemma}

\begin{consequence}[Název důkazu]
Abcd. Abcd. Abcd. Abcd. Abcd. Abcd. Abcd. Abcd. Abcd. Abcd. Abcd. Abcd. Abcd. Abcd. Abcd. Abcd. Abcd. Abcd. Abcd. Abcd. Abcd. Abcd. Abcd. Abcd. Abcd. Abcd. Abcd. Abcd. Abcd. 
\end{consequence}

\begin{theorem}[Pumpovací věta]
Abcd. Abcd. Abcd. Abcd. Abcd. Abcd. Abcd. Abcd. Abcd. Abcd. Abcd. Abcd. Abcd. Abcd. Abcd. Abcd. Abcd. Abcd. Abcd. Abcd. Abcd. Abcd. Abcd. Abcd. Abcd. Abcd. Abcd. Abcd. Abcd. Abcd. 
\end{theorem}


\begin{kicode}{cpp}{kod:cpp}{\cpp}
int main("cs acsa") // komentar
int main("cs acsa") // komentar
int main("cs acsa") // komentar
int main("cs acsa") // komentar
int main("cs acsa") // komentar
\end{kicode}

\begin{kicode}{JavaScript}{}{JS}
new object() // komentar
\end{kicode}

\begin{kicode}{csharp}{}{\csharp}
public static int main("cs acsa") // komentar
\end{kicode}

\begin{kicode}{SQL}{}{SQL}
SELECT * FROM table_1; /* komentar */
\end{kicode}

\begin{kicode}{TutorialD}{}{TutorialD}
table_1 AND table_2;
\end{kicode}

%% Závěry práce. V jazyce práce a anglicky. Text pro jiný než
%% nastavený jazyk práce (nepovinným parametrem language makra
%% \documentclass, výchozí český) se zadává použitím makra s uvedením
%% jazyka jako nepovinného parametru.
\begin{kiconclusions}
Závěr práce v \uv{českém} jazyce.
\end{kiconclusions}

\begin{kiconclusions}[english]
Thesis conclusions in \uv{English}.
\end{kiconclusions}

%% Přílohy obsahu textu práce, za makrem \appendix.
\appendix

\section{První příloha}
Text první přílohy

\section{Druhá příloha}
Text druhé přílohy

%% Obsah přiloženého CD/DVD. Poslední příloha. Upravte podle vlastní
%% práce!
\section{Obsah přiloženého CD/DVD} \label{sec:ObsahCD}

Na samotném konci textu práce je uveden stručný popis obsahu
přiloženého CD/DVD, tj.~jeho závazné adresářové struktury, důležitých
souborů apod.

\begin{description}

\item[\texttt{bin/}] \hfill \\
  Instalátor \textsc{Instalator} programu, popř.~program
  \textsc{Program}, spustitelné přímo z~CD/DVD. / Kompletní adresářová
  struktura webové aplikace \textsc{Webovka} (v~ZIP archivu) pro
  zkopírování na webový server. Adresář obsahuje i~všechny runtime
  knihovny a~další soubory potřebné pro bezproblémový běh instalátoru
  a~programu z~CD/DVD / pro bezproblémový provoz webové aplikace na
  webovém serveru.

\item[\texttt{doc/}] \hfill \\
  Text práce ve formátu PDF, vytvořený s~použitím závazného stylu KI
  PřF UP v~Olomouci pro závěrečné práce, včetně všech příloh,
  a~všechny soubory potřebné pro bezproblémové vygenerování PDF
  dokumentu textu (v~ZIP archivu), tj.~zdrojový text textu, vložené
  obrázky, apod.

\item[\texttt{src/}] \hfill \\
  Kompletní zdrojové texty programu \textsc{Program} / webové aplikace
  \textsc{Webovka} se všemi potřebnými (příp.~převzatými) zdrojovými
  texty, knihovnami a~dalšími soubory potřebnými pro bezproblémové
  vytvoření spustitelných verzí programu / adresářové struktury pro
  zkopírování na webový server.

\item[\texttt{readme.txt}] \hfill \\
  Instrukce pro instalaci a~spuštění programu \textsc{Program}, včetně
  všech požadavků pro jeho bezproblémový provoz. / Instrukce pro
  nasazení webové aplikace \textsc{Webovka} na webový server, včetně
  všech požadavků pro její bezproblémový provoz, a~webová adresa, na
  které je aplikace nasazena pro účel testování při tvorbě posudků
  práce a~pro účel obhajoby práce.

\end{description}

Navíc CD/DVD obsahuje:

\begin{description}

\item[\texttt{data/}] \hfill \\
  Ukázková a~testovací data použitá v~práci a~pro potřeby testování
  práce při tvorbě posudků a~obhajoby práce.

\item[\texttt{install/}] \hfill \\
  Instalátory aplikací, runtime knihoven a~jiných souborů potřebných
  pro provoz programu \textsc{Program} / webové aplikace
  \textsc{Webovka}, které nejsou standardní součástí operačního
  systému určeného pro běh programu / provoz webové aplikace.

\item[\texttt{literature/}] \hfill \\
  Vybrané položky bibliografie, příp.~jiná užitečná literatura
  vztahující se k~práci.

\end{description}

U~veškerých cizích převzatých materiálů obsažených na CD/DVD jejich
zahrnutí dovolují podmínky pro jejich šíření nebo přiložený souhlas
držitele copyrightu. Pro všechny použité (a~citované) materiály,
u~kterých toto není splněno a~nejsou tak obsaženy na CD/DVD, je uveden
jejich zdroj (např.~webová adresa) v~bibliografii nebo textu práce
nebo v souboru \texttt{readme.txt}.

%% -------------------------------------------------------------------

%% Sazba volitelného seznamu zkratek, za přílohami.
\printglossary

%% Sazba povinné bibliografie, za přílohami (případně i za seznamem
%% zkratek). Při použití BibLaTeXu použijte makro
%% \printbibliography. jinak prostředí thebibliography. Ne obojí!

%% Sazba i v textu necitovaných zdrojů, při použití
%% BibLaTeXu. Volitelné.
\nocite{*}
%% Vlastní sazba bibliografie při použití BibLaTeXu.
\printbibliography

%% Bibliografie, včetně sazby, při nepoužití BibLaTeXu.
% \begin{thebibliography}{9}
%\bibitem{kniha2} \uppercase{Hawke}, Paul. NanoHttpd: Light-weight HTTP server designed for embedding in other applications. GitHub [online]. 2014-05-12. [cit. 2014-12-06]. Dostupné z: \url{https://github.com/NanoHttpd/nanohttpd}
%
%\bibitem{jeske13} \uppercase{Jeske}, David; \uppercase{Novák}, Josef. Simple HTTP Server in \csharp: Threaded synchronous HTTP Server abstract class, to respond to HTTP requests. CodeProject: For those who code [online]. 2014-05-24. [cit. 2014-12-06]. Dostupné z: \url{http://www.codeproject.com/Articles/137979/Simple-HTTP-Server-in-C}
%
%\bibitem{uzis2012} \uppercase{ÚSTAV ZDRAVOTNICKÝCH INFORMACÍ A STATISTIKY ČR}. Lékaři, zubní lékaři a farmaceuti 2012 [online]. Praha 2, Palackého náměstí 4: Ústav zdravotnických informací a statistiky ČR, 2012 [cit. 2014-12-06]. ISBN 978-80-7472-089-5. Dostupné z: \url{http://www.uzis.cz/publikace/lekari-zubni-lekari-farmaceuti-2012}
% \end{thebibliography}

%% Sazba volitelného rejstříku, za bibliografií.
\printindex

\end{document}
